\chapter{Podsumowanie}
\section{Plany rozbudowy aplikacji}

Po zakończeniu pierwszej fazy projektu "Sklep", jedną ze ścieżek na rozwój aplikacji jest wprowadzenie nowych funkcjonalności oraz usprawnień mających na celu zwiększenie atrakcyjności, użyteczności i efektywności systemu.\newline

Aktualny system oparty jest na interfejsie konsolowym, jednak w planach jest wprowadzenie interfejsu graficznego (GUI), co znacznie poprawi wrażenia użytkownika. Przejście na interfejs z grafiką umożliwi bardziej intuicyjną nawigację oraz lepszą prezentację produktów. Jedną z możliwości jest stworzenie aplikacji okienkowej w rozwiązaniu WPF. Przy zmianie na tego typu aplikację, logika biznesowa pozostanie bez zmian. Trzeba ją jednak będzie dopasować do nowego interfejsu graficznego.\newline

Kolejną możliwością rozwoju jest przejście na wiele kategorii sprzedawanych produktów zwiększając asortyment lub podzial istniejących produktów na kategorie. Ułatwi to użytkownikom wyszukiwanie produktów oraz uporządkuje produkty w aplikacji.\newline

Podążając za trendem zaobserwowanym w innych aplikacjach tego typu, można utworzyć mechanizm systemu oceny i recenzji. Dodanie funkcji, która umożliwi klientom wystawianie ocen i recenzji produktów, nie tylko zwiększy zaufanie do sklepu, ale także dostarczy istotnych informacji zwrotnych dla sprzedawcy. Na podstawie tych danych system będzie mógł wyświetlać spersonalizowane sugestie produktów i je reklamować.

\section{Podsumowanie zrealizowanych prac}

Podczas tworzenia projektu "Sklep" w języku C\# w postaci programu konsolowego, głównym celem było stworzenie kompleksowego systemu obsługującego zarówno perspektywę sprzedawcy, jak i klienta. Dzięki zaimplementowanym funkcjonalnościom, aplikacja umożliwia sprawną obsługę procesów biznesowych, usprawnia zarządzanie asortymentem oraz zapewnia przyjazne i intuicyjne środowisko dla klientów dokonujących zakupów. Prace zostały zakończone sukcesem, a uzyskane rezultaty odpowiadają stawianym przed projektem celom.


